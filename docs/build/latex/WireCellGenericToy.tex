%% Generated by Sphinx.
\def\sphinxdocclass{report}
\documentclass[letterpaper,10pt,english]{sphinxmanual}
\ifdefined\pdfpxdimen
   \let\sphinxpxdimen\pdfpxdimen\else\newdimen\sphinxpxdimen
\fi \sphinxpxdimen=.75bp\relax

\PassOptionsToPackage{warn}{textcomp}
\usepackage[utf8]{inputenc}
\ifdefined\DeclareUnicodeCharacter
 \ifdefined\DeclareUnicodeCharacterAsOptional
  \DeclareUnicodeCharacter{"00A0}{\nobreakspace}
  \DeclareUnicodeCharacter{"2500}{\sphinxunichar{2500}}
  \DeclareUnicodeCharacter{"2502}{\sphinxunichar{2502}}
  \DeclareUnicodeCharacter{"2514}{\sphinxunichar{2514}}
  \DeclareUnicodeCharacter{"251C}{\sphinxunichar{251C}}
  \DeclareUnicodeCharacter{"2572}{\textbackslash}
 \else
  \DeclareUnicodeCharacter{00A0}{\nobreakspace}
  \DeclareUnicodeCharacter{2500}{\sphinxunichar{2500}}
  \DeclareUnicodeCharacter{2502}{\sphinxunichar{2502}}
  \DeclareUnicodeCharacter{2514}{\sphinxunichar{2514}}
  \DeclareUnicodeCharacter{251C}{\sphinxunichar{251C}}
  \DeclareUnicodeCharacter{2572}{\textbackslash}
 \fi
\fi
\usepackage{cmap}
\usepackage[T1]{fontenc}
\usepackage{amsmath,amssymb,amstext}
\usepackage{babel}
\usepackage{times}
\usepackage[Bjarne]{fncychap}
\usepackage{sphinx}

\usepackage{geometry}

% Include hyperref last.
\usepackage{hyperref}
% Fix anchor placement for figures with captions.
\usepackage{hypcap}% it must be loaded after hyperref.
% Set up styles of URL: it should be placed after hyperref.
\urlstyle{same}
\addto\captionsenglish{\renewcommand{\contentsname}{Contents:}}

\addto\captionsenglish{\renewcommand{\figurename}{Fig.}}
\addto\captionsenglish{\renewcommand{\tablename}{Table}}
\addto\captionsenglish{\renewcommand{\literalblockname}{Listing}}

\addto\captionsenglish{\renewcommand{\literalblockcontinuedname}{continued from previous page}}
\addto\captionsenglish{\renewcommand{\literalblockcontinuesname}{continues on next page}}

\addto\extrasenglish{\def\pageautorefname{page}}

\setcounter{tocdepth}{1}



\title{Wire Cell Generic Toy Documentation}
\date{Aug 19, 2018}
\release{1.0}
\author{Orgho Neogi}
\newcommand{\sphinxlogo}{\vbox{}}
\renewcommand{\releasename}{Release}
\makeindex

\begin{document}

\maketitle
\sphinxtableofcontents
\phantomsection\label{\detokenize{index::doc}}



\chapter{wireCellGenericToy}
\label{\detokenize{modules:wirecellgenerictoy}}\label{\detokenize{modules::doc}}

\section{dataTypes module}
\label{\detokenize{dataTypes:module-dataTypes}}\label{\detokenize{dataTypes:datatypes-module}}\label{\detokenize{dataTypes::doc}}\index{dataTypes (module)}\index{Blob (class in dataTypes)}

\begin{fulllineitems}
\phantomsection\label{\detokenize{dataTypes:dataTypes.Blob}}\pysigline{\sphinxbfcode{\sphinxupquote{class }}\sphinxcode{\sphinxupquote{dataTypes.}}\sphinxbfcode{\sphinxupquote{Blob}}}
Bases: \sphinxcode{\sphinxupquote{tuple}}

Blob detected in tomographic slice
\begin{quote}\begin{description}
\item[{Attributes}] \leavevmode\begin{description}
\item[{{\hyperref[\detokenize{dataTypes:dataTypes.Blob.charge}]{\sphinxcrossref{\sphinxcode{\sphinxupquote{charge}}}}}}] \leavevmode
Charge Associated with the Blob

\item[{{\hyperref[\detokenize{dataTypes:dataTypes.Blob.points}]{\sphinxcrossref{\sphinxcode{\sphinxupquote{points}}}}}}] \leavevmode
Points that define the ConvexHull of the blob

\item[{{\hyperref[\detokenize{dataTypes:dataTypes.Blob.wires}]{\sphinxcrossref{\sphinxcode{\sphinxupquote{wires}}}}}}] \leavevmode
Wires that bind the Blob

\end{description}

\end{description}\end{quote}
\paragraph{Methods}


\begin{savenotes}\sphinxatlongtablestart\begin{longtable}{\X{1}{2}\X{1}{2}}
\hline

\endfirsthead

\multicolumn{2}{c}%
{\makebox[0pt]{\sphinxtablecontinued{\tablename\ \thetable{} -- continued from previous page}}}\\
\hline

\endhead

\hline
\multicolumn{2}{r}{\makebox[0pt][r]{\sphinxtablecontinued{Continued on next page}}}\\
\endfoot

\endlastfoot

\sphinxcode{\sphinxupquote{count}}(\$self, value, /)
&
Return number of occurrences of value.
\\
\hline
\sphinxcode{\sphinxupquote{index}}(\$self, value{[}, start, stop{]})
&
Return first index of value.
\\
\hline
\end{longtable}\sphinxatlongtableend\end{savenotes}
\index{charge (dataTypes.Blob attribute)}

\begin{fulllineitems}
\phantomsection\label{\detokenize{dataTypes:dataTypes.Blob.charge}}\pysigline{\sphinxbfcode{\sphinxupquote{charge}}}
Charge Associated with the Blob

\end{fulllineitems}

\index{points (dataTypes.Blob attribute)}

\begin{fulllineitems}
\phantomsection\label{\detokenize{dataTypes:dataTypes.Blob.points}}\pysigline{\sphinxbfcode{\sphinxupquote{points}}}
Points that define the ConvexHull of the blob

\end{fulllineitems}

\index{wires (dataTypes.Blob attribute)}

\begin{fulllineitems}
\phantomsection\label{\detokenize{dataTypes:dataTypes.Blob.wires}}\pysigline{\sphinxbfcode{\sphinxupquote{wires}}}
Wires that bind the Blob

\end{fulllineitems}


\end{fulllineitems}

\index{Cell (class in dataTypes)}

\begin{fulllineitems}
\phantomsection\label{\detokenize{dataTypes:dataTypes.Cell}}\pysigline{\sphinxbfcode{\sphinxupquote{class }}\sphinxcode{\sphinxupquote{dataTypes.}}\sphinxbfcode{\sphinxupquote{Cell}}}
Bases: \sphinxcode{\sphinxupquote{tuple}}

Merged Cell in Detector
\begin{quote}\begin{description}
\item[{Attributes}] \leavevmode\begin{description}
\item[{{\hyperref[\detokenize{dataTypes:dataTypes.Cell.points}]{\sphinxcrossref{\sphinxcode{\sphinxupquote{points}}}}}}] \leavevmode
points binding the Cell

\item[{{\hyperref[\detokenize{dataTypes:dataTypes.Cell.wires}]{\sphinxcrossref{\sphinxcode{\sphinxupquote{wires}}}}}}] \leavevmode
binding wires of the Cell

\end{description}

\end{description}\end{quote}
\paragraph{Methods}


\begin{savenotes}\sphinxatlongtablestart\begin{longtable}{\X{1}{2}\X{1}{2}}
\hline

\endfirsthead

\multicolumn{2}{c}%
{\makebox[0pt]{\sphinxtablecontinued{\tablename\ \thetable{} -- continued from previous page}}}\\
\hline

\endhead

\hline
\multicolumn{2}{r}{\makebox[0pt][r]{\sphinxtablecontinued{Continued on next page}}}\\
\endfoot

\endlastfoot

\sphinxcode{\sphinxupquote{count}}(\$self, value, /)
&
Return number of occurrences of value.
\\
\hline
\sphinxcode{\sphinxupquote{index}}(\$self, value{[}, start, stop{]})
&
Return first index of value.
\\
\hline
\end{longtable}\sphinxatlongtableend\end{savenotes}
\index{points (dataTypes.Cell attribute)}

\begin{fulllineitems}
\phantomsection\label{\detokenize{dataTypes:dataTypes.Cell.points}}\pysigline{\sphinxbfcode{\sphinxupquote{points}}}
points binding the Cell

\end{fulllineitems}

\index{wires (dataTypes.Cell attribute)}

\begin{fulllineitems}
\phantomsection\label{\detokenize{dataTypes:dataTypes.Cell.wires}}\pysigline{\sphinxbfcode{\sphinxupquote{wires}}}
binding wires of the Cell

\end{fulllineitems}


\end{fulllineitems}

\index{DetectorVolume (class in dataTypes)}

\begin{fulllineitems}
\phantomsection\label{\detokenize{dataTypes:dataTypes.DetectorVolume}}\pysigline{\sphinxbfcode{\sphinxupquote{class }}\sphinxcode{\sphinxupquote{dataTypes.}}\sphinxbfcode{\sphinxupquote{DetectorVolume}}}
Bases: \sphinxcode{\sphinxupquote{tuple}}

2D dimensions of the detector
\begin{quote}\begin{description}
\item[{Attributes}] \leavevmode\begin{description}
\item[{{\hyperref[\detokenize{dataTypes:dataTypes.DetectorVolume.height}]{\sphinxcrossref{\sphinxcode{\sphinxupquote{height}}}}}}] \leavevmode
height of detector

\item[{{\hyperref[\detokenize{dataTypes:dataTypes.DetectorVolume.width}]{\sphinxcrossref{\sphinxcode{\sphinxupquote{width}}}}}}] \leavevmode
width of detector

\end{description}

\end{description}\end{quote}
\paragraph{Methods}


\begin{savenotes}\sphinxatlongtablestart\begin{longtable}{\X{1}{2}\X{1}{2}}
\hline

\endfirsthead

\multicolumn{2}{c}%
{\makebox[0pt]{\sphinxtablecontinued{\tablename\ \thetable{} -- continued from previous page}}}\\
\hline

\endhead

\hline
\multicolumn{2}{r}{\makebox[0pt][r]{\sphinxtablecontinued{Continued on next page}}}\\
\endfoot

\endlastfoot

\sphinxcode{\sphinxupquote{count}}(\$self, value, /)
&
Return number of occurrences of value.
\\
\hline
\sphinxcode{\sphinxupquote{index}}(\$self, value{[}, start, stop{]})
&
Return first index of value.
\\
\hline
\end{longtable}\sphinxatlongtableend\end{savenotes}
\index{height (dataTypes.DetectorVolume attribute)}

\begin{fulllineitems}
\phantomsection\label{\detokenize{dataTypes:dataTypes.DetectorVolume.height}}\pysigline{\sphinxbfcode{\sphinxupquote{height}}}
height of detector

\end{fulllineitems}

\index{width (dataTypes.DetectorVolume attribute)}

\begin{fulllineitems}
\phantomsection\label{\detokenize{dataTypes:dataTypes.DetectorVolume.width}}\pysigline{\sphinxbfcode{\sphinxupquote{width}}}
width of detector

\end{fulllineitems}


\end{fulllineitems}

\index{Line (class in dataTypes)}

\begin{fulllineitems}
\phantomsection\label{\detokenize{dataTypes:dataTypes.Line}}\pysigline{\sphinxbfcode{\sphinxupquote{class }}\sphinxcode{\sphinxupquote{dataTypes.}}\sphinxbfcode{\sphinxupquote{Line}}}
Bases: \sphinxcode{\sphinxupquote{tuple}}

A line defined by two points
\begin{quote}\begin{description}
\item[{Attributes}] \leavevmode\begin{description}
\item[{{\hyperref[\detokenize{dataTypes:dataTypes.Line.point0}]{\sphinxcrossref{\sphinxcode{\sphinxupquote{point0}}}}}}] \leavevmode
First Point

\item[{{\hyperref[\detokenize{dataTypes:dataTypes.Line.point1}]{\sphinxcrossref{\sphinxcode{\sphinxupquote{point1}}}}}}] \leavevmode
Second Point

\end{description}

\end{description}\end{quote}
\paragraph{Methods}


\begin{savenotes}\sphinxatlongtablestart\begin{longtable}{\X{1}{2}\X{1}{2}}
\hline

\endfirsthead

\multicolumn{2}{c}%
{\makebox[0pt]{\sphinxtablecontinued{\tablename\ \thetable{} -- continued from previous page}}}\\
\hline

\endhead

\hline
\multicolumn{2}{r}{\makebox[0pt][r]{\sphinxtablecontinued{Continued on next page}}}\\
\endfoot

\endlastfoot

\sphinxcode{\sphinxupquote{count}}(\$self, value, /)
&
Return number of occurrences of value.
\\
\hline
\sphinxcode{\sphinxupquote{index}}(\$self, value{[}, start, stop{]})
&
Return first index of value.
\\
\hline
\end{longtable}\sphinxatlongtableend\end{savenotes}
\index{point0 (dataTypes.Line attribute)}

\begin{fulllineitems}
\phantomsection\label{\detokenize{dataTypes:dataTypes.Line.point0}}\pysigline{\sphinxbfcode{\sphinxupquote{point0}}}
First Point

\end{fulllineitems}

\index{point1 (dataTypes.Line attribute)}

\begin{fulllineitems}
\phantomsection\label{\detokenize{dataTypes:dataTypes.Line.point1}}\pysigline{\sphinxbfcode{\sphinxupquote{point1}}}
Second Point

\end{fulllineitems}


\end{fulllineitems}

\index{PlaneInfo (class in dataTypes)}

\begin{fulllineitems}
\phantomsection\label{\detokenize{dataTypes:dataTypes.PlaneInfo}}\pysigline{\sphinxbfcode{\sphinxupquote{class }}\sphinxcode{\sphinxupquote{dataTypes.}}\sphinxbfcode{\sphinxupquote{PlaneInfo}}}
Bases: \sphinxcode{\sphinxupquote{tuple}}

List of information about planes
\begin{quote}\begin{description}
\item[{Attributes}] \leavevmode\begin{description}
\item[{{\hyperref[\detokenize{dataTypes:dataTypes.PlaneInfo.angle}]{\sphinxcrossref{\sphinxcode{\sphinxupquote{angle}}}}}}] \leavevmode
angle by whitch plane is roatated

\item[{{\hyperref[\detokenize{dataTypes:dataTypes.PlaneInfo.cos}]{\sphinxcrossref{\sphinxcode{\sphinxupquote{cos}}}}}}] \leavevmode
cos of the plane angle

\item[{{\hyperref[\detokenize{dataTypes:dataTypes.PlaneInfo.gradient}]{\sphinxcrossref{\sphinxcode{\sphinxupquote{gradient}}}}}}] \leavevmode
gradient of the wires in plane

\item[{{\hyperref[\detokenize{dataTypes:dataTypes.PlaneInfo.noOfWires}]{\sphinxcrossref{\sphinxcode{\sphinxupquote{noOfWires}}}}}}] \leavevmode
number of wires in the plane

\item[{{\hyperref[\detokenize{dataTypes:dataTypes.PlaneInfo.originTranslation}]{\sphinxcrossref{\sphinxcode{\sphinxupquote{originTranslation}}}}}}] \leavevmode
x offset for origin of the plane (y origin is always 0)

\item[{{\hyperref[\detokenize{dataTypes:dataTypes.PlaneInfo.pitch}]{\sphinxcrossref{\sphinxcode{\sphinxupquote{pitch}}}}}}] \leavevmode
wire pitch for the plane

\item[{{\hyperref[\detokenize{dataTypes:dataTypes.PlaneInfo.sin}]{\sphinxcrossref{\sphinxcode{\sphinxupquote{sin}}}}}}] \leavevmode
sin of the plane angle

\end{description}

\end{description}\end{quote}
\paragraph{Methods}


\begin{savenotes}\sphinxatlongtablestart\begin{longtable}{\X{1}{2}\X{1}{2}}
\hline

\endfirsthead

\multicolumn{2}{c}%
{\makebox[0pt]{\sphinxtablecontinued{\tablename\ \thetable{} -- continued from previous page}}}\\
\hline

\endhead

\hline
\multicolumn{2}{r}{\makebox[0pt][r]{\sphinxtablecontinued{Continued on next page}}}\\
\endfoot

\endlastfoot

\sphinxcode{\sphinxupquote{count}}(\$self, value, /)
&
Return number of occurrences of value.
\\
\hline
\sphinxcode{\sphinxupquote{index}}(\$self, value{[}, start, stop{]})
&
Return first index of value.
\\
\hline
\end{longtable}\sphinxatlongtableend\end{savenotes}
\index{angle (dataTypes.PlaneInfo attribute)}

\begin{fulllineitems}
\phantomsection\label{\detokenize{dataTypes:dataTypes.PlaneInfo.angle}}\pysigline{\sphinxbfcode{\sphinxupquote{angle}}}
angle by whitch plane is roatated

\end{fulllineitems}

\index{cos (dataTypes.PlaneInfo attribute)}

\begin{fulllineitems}
\phantomsection\label{\detokenize{dataTypes:dataTypes.PlaneInfo.cos}}\pysigline{\sphinxbfcode{\sphinxupquote{cos}}}
cos of the plane angle

\end{fulllineitems}

\index{gradient (dataTypes.PlaneInfo attribute)}

\begin{fulllineitems}
\phantomsection\label{\detokenize{dataTypes:dataTypes.PlaneInfo.gradient}}\pysigline{\sphinxbfcode{\sphinxupquote{gradient}}}
gradient of the wires in plane

\end{fulllineitems}

\index{noOfWires (dataTypes.PlaneInfo attribute)}

\begin{fulllineitems}
\phantomsection\label{\detokenize{dataTypes:dataTypes.PlaneInfo.noOfWires}}\pysigline{\sphinxbfcode{\sphinxupquote{noOfWires}}}
number of wires in the plane

\end{fulllineitems}

\index{originTranslation (dataTypes.PlaneInfo attribute)}

\begin{fulllineitems}
\phantomsection\label{\detokenize{dataTypes:dataTypes.PlaneInfo.originTranslation}}\pysigline{\sphinxbfcode{\sphinxupquote{originTranslation}}}
x offset for origin of the plane (y origin is always 0)

\end{fulllineitems}

\index{pitch (dataTypes.PlaneInfo attribute)}

\begin{fulllineitems}
\phantomsection\label{\detokenize{dataTypes:dataTypes.PlaneInfo.pitch}}\pysigline{\sphinxbfcode{\sphinxupquote{pitch}}}
wire pitch for the plane

\end{fulllineitems}

\index{sin (dataTypes.PlaneInfo attribute)}

\begin{fulllineitems}
\phantomsection\label{\detokenize{dataTypes:dataTypes.PlaneInfo.sin}}\pysigline{\sphinxbfcode{\sphinxupquote{sin}}}
sin of the plane angle

\end{fulllineitems}


\end{fulllineitems}

\index{Point (class in dataTypes)}

\begin{fulllineitems}
\phantomsection\label{\detokenize{dataTypes:dataTypes.Point}}\pysigline{\sphinxbfcode{\sphinxupquote{class }}\sphinxcode{\sphinxupquote{dataTypes.}}\sphinxbfcode{\sphinxupquote{Point}}}
Bases: \sphinxcode{\sphinxupquote{tuple}}

A 2-dimensional coordinate
\begin{quote}\begin{description}
\item[{Attributes}] \leavevmode\begin{description}
\item[{{\hyperref[\detokenize{dataTypes:dataTypes.Point.x}]{\sphinxcrossref{\sphinxcode{\sphinxupquote{x}}}}}}] \leavevmode
x coordinate

\item[{{\hyperref[\detokenize{dataTypes:dataTypes.Point.y}]{\sphinxcrossref{\sphinxcode{\sphinxupquote{y}}}}}}] \leavevmode
y coordinate

\end{description}

\end{description}\end{quote}
\paragraph{Methods}


\begin{savenotes}\sphinxatlongtablestart\begin{longtable}{\X{1}{2}\X{1}{2}}
\hline

\endfirsthead

\multicolumn{2}{c}%
{\makebox[0pt]{\sphinxtablecontinued{\tablename\ \thetable{} -- continued from previous page}}}\\
\hline

\endhead

\hline
\multicolumn{2}{r}{\makebox[0pt][r]{\sphinxtablecontinued{Continued on next page}}}\\
\endfoot

\endlastfoot

\sphinxcode{\sphinxupquote{count}}(\$self, value, /)
&
Return number of occurrences of value.
\\
\hline
\sphinxcode{\sphinxupquote{index}}(\$self, value{[}, start, stop{]})
&
Return first index of value.
\\
\hline
\end{longtable}\sphinxatlongtableend\end{savenotes}
\index{x (dataTypes.Point attribute)}

\begin{fulllineitems}
\phantomsection\label{\detokenize{dataTypes:dataTypes.Point.x}}\pysigline{\sphinxbfcode{\sphinxupquote{x}}}
x coordinate

\end{fulllineitems}

\index{y (dataTypes.Point attribute)}

\begin{fulllineitems}
\phantomsection\label{\detokenize{dataTypes:dataTypes.Point.y}}\pysigline{\sphinxbfcode{\sphinxupquote{y}}}
y coordinate

\end{fulllineitems}


\end{fulllineitems}



\section{utilities module}
\label{\detokenize{utilities:module-utilities}}\label{\detokenize{utilities:utilities-module}}\label{\detokenize{utilities::doc}}\index{utilities (module)}\index{fireWires() (in module utilities)}

\begin{fulllineitems}
\phantomsection\label{\detokenize{utilities:utilities.fireWires}}\pysiglinewithargsret{\sphinxcode{\sphinxupquote{utilities.}}\sphinxbfcode{\sphinxupquote{fireWires}}}{\emph{planes}, \emph{points}}{}
Show which wires have been hit for a given blob
\begin{quote}\begin{description}
\item[{Parameters}] \leavevmode\begin{description}
\item[{\sphinxstylestrong{planes}}] \leavevmode{[}list of PlaneInfo{]}
A list containing information for all the planes in the detector

\item[{\sphinxstylestrong{points}}] \leavevmode{[}list of points{]}
points defining ConvexHull of blob

\end{description}

\item[{Returns}] \leavevmode\begin{description}
\item[{\sphinxstylestrong{2d list of int}}] \leavevmode
A list that has lists of merged wire numbers. one list for every plane

\end{description}

\end{description}\end{quote}

\end{fulllineitems}

\index{generatePlaneInfo() (in module utilities)}

\begin{fulllineitems}
\phantomsection\label{\detokenize{utilities:utilities.generatePlaneInfo}}\pysiglinewithargsret{\sphinxcode{\sphinxupquote{utilities.}}\sphinxbfcode{\sphinxupquote{generatePlaneInfo}}}{\emph{wirePitches}, \emph{volume}, \emph{angles}}{}
Generates information about planes based on wire pitches and volume assuming equal angle between each plane and the next
\begin{quote}\begin{description}
\item[{Parameters}] \leavevmode\begin{description}
\item[{\sphinxstylestrong{wirePitches}}] \leavevmode{[}list or tuple of int or float{]}
A list of wire pitches, each number corresponding to the pitch of each plane

\item[{\sphinxstylestrong{volume}}] \leavevmode{[}DetectorVolume{]}
The width and height of the detector

\item[{\sphinxstylestrong{angles}}] \leavevmode{[}list or tuple of float{]}
List of the angles of the wire plane in radians

\end{description}

\item[{Returns}] \leavevmode\begin{description}
\item[{\sphinxstylestrong{list of PlaneInfo}}] \leavevmode
List of information regarding the planes.

\end{description}

\end{description}\end{quote}

\end{fulllineitems}

\index{getChannelNo() (in module utilities)}

\begin{fulllineitems}
\phantomsection\label{\detokenize{utilities:utilities.getChannelNo}}\pysiglinewithargsret{\sphinxcode{\sphinxupquote{utilities.}}\sphinxbfcode{\sphinxupquote{getChannelNo}}}{\emph{planes}, \emph{wireNo}, \emph{planeNo}}{}
Get Channel Number for a merged wire
\begin{quote}\begin{description}
\item[{Parameters}] \leavevmode\begin{description}
\item[{\sphinxstylestrong{planes}}] \leavevmode{[}list of PlaneInfo{]}
A list containing information for all the planes in the detector

\item[{\sphinxstylestrong{wireNo}}] \leavevmode{[}int{]}
Primitive Wire

\item[{\sphinxstylestrong{planeNo}}] \leavevmode{[}int{]}
Index of plane the wires are in

\end{description}

\item[{Returns}] \leavevmode\begin{description}
\item[{\sphinxstylestrong{int}}] \leavevmode
Channel number for a primitive wire

\end{description}

\end{description}\end{quote}

\end{fulllineitems}

\index{pointInWire() (in module utilities)}

\begin{fulllineitems}
\phantomsection\label{\detokenize{utilities:utilities.pointInWire}}\pysiglinewithargsret{\sphinxcode{\sphinxupquote{utilities.}}\sphinxbfcode{\sphinxupquote{pointInWire}}}{\emph{plane}, \emph{point}, \emph{wire}}{}
Check if a point is inside a certain merged wire
\begin{quote}\begin{description}
\item[{Parameters}] \leavevmode\begin{description}
\item[{\sphinxstylestrong{plane}}] \leavevmode{[}PlaneInfo{]}
Plane information for the plane the wire number is for

\item[{\sphinxstylestrong{point}}] \leavevmode{[}Point{]}
The point being queried

\item[{\sphinxstylestrong{wire}}] \leavevmode{[}tuple{[}2{]} of int{]}
A merged wire

\end{description}

\item[{Returns}] \leavevmode\begin{description}
\item[{\sphinxstylestrong{Boolean}}] \leavevmode
True if point is in wire, False if not

\end{description}

\end{description}\end{quote}

\end{fulllineitems}

\index{wireNumberFromPoint() (in module utilities)}

\begin{fulllineitems}
\phantomsection\label{\detokenize{utilities:utilities.wireNumberFromPoint}}\pysiglinewithargsret{\sphinxcode{\sphinxupquote{utilities.}}\sphinxbfcode{\sphinxupquote{wireNumberFromPoint}}}{\emph{plane}, \emph{point}}{}
Generates the wire number for a point given a certain plane
\begin{quote}\begin{description}
\item[{Parameters}] \leavevmode\begin{description}
\item[{\sphinxstylestrong{plane}}] \leavevmode{[}PlaneInfo{]}
Plane information for the plane the wire number is for

\item[{\sphinxstylestrong{point}}] \leavevmode{[}Point{]}
The point being queried

\end{description}

\item[{Returns}] \leavevmode\begin{description}
\item[{\sphinxstylestrong{int}}] \leavevmode
Primitive Wire number

\end{description}

\end{description}\end{quote}

\end{fulllineitems}



\section{geometryGen module}
\label{\detokenize{geometryGen:module-geometryGen}}\label{\detokenize{geometryGen:geometrygen-module}}\label{\detokenize{geometryGen::doc}}\index{geometryGen (module)}\index{createMergedWires() (in module geometryGen)}

\begin{fulllineitems}
\phantomsection\label{\detokenize{geometryGen:geometryGen.createMergedWires}}\pysiglinewithargsret{\sphinxcode{\sphinxupquote{geometryGen.}}\sphinxbfcode{\sphinxupquote{createMergedWires}}}{\emph{firedWirePrimitives}}{}
Merged wires from a list of fired wire primitives
\begin{quote}\begin{description}
\item[{Parameters}] \leavevmode\begin{description}
\item[{\sphinxstylestrong{firedWirePrimitives}}] \leavevmode{[}list of int{]}
List of primitive wire numberss

\end{description}

\item[{Returns}] \leavevmode\begin{description}
\item[{\sphinxstylestrong{tuple{[}2{]} of int}}] \leavevmode
Merged Wire

\end{description}

\end{description}\end{quote}

\end{fulllineitems}

\index{generateBlobs() (in module geometryGen)}

\begin{fulllineitems}
\phantomsection\label{\detokenize{geometryGen:geometryGen.generateBlobs}}\pysiglinewithargsret{\sphinxcode{\sphinxupquote{geometryGen.}}\sphinxbfcode{\sphinxupquote{generateBlobs}}}{\emph{planes}, \emph{volume}}{}
Generate random true blobs
\begin{quote}\begin{description}
\item[{Parameters}] \leavevmode\begin{description}
\item[{\sphinxstylestrong{planes}}] \leavevmode{[}list of PlaneInfo{]}
A list containing information for all the planes in the detector

\item[{\sphinxstylestrong{volume}}] \leavevmode{[}DetectorVolume{]}
The width and height of the detector

\end{description}

\item[{Returns}] \leavevmode\begin{description}
\item[{\sphinxstylestrong{list of Blob}}] \leavevmode
A list of all the blobs in the event

\end{description}

\end{description}\end{quote}

\end{fulllineitems}

\index{generateEvent() (in module geometryGen)}

\begin{fulllineitems}
\phantomsection\label{\detokenize{geometryGen:geometryGen.generateEvent}}\pysiglinewithargsret{\sphinxcode{\sphinxupquote{geometryGen.}}\sphinxbfcode{\sphinxupquote{generateEvent}}}{\emph{planes}, \emph{blobs}}{}
Generate List of wire primitives that were fired based on true blobs
\begin{quote}\begin{description}
\item[{Parameters}] \leavevmode\begin{description}
\item[{\sphinxstylestrong{planes}}] \leavevmode{[}list of PlaneInfo{]}
A list containing information for all the planes in the detector

\item[{\sphinxstylestrong{blobs}}] \leavevmode{[}list of Blob{]}
A list containing true blobs in event

\end{description}

\item[{Returns}] \leavevmode\begin{description}
\item[{\sphinxstylestrong{list of list of ints}}] \leavevmode
Primitive fired wires from every plane

\end{description}

\end{description}\end{quote}

\end{fulllineitems}

\index{mergeEvent() (in module geometryGen)}

\begin{fulllineitems}
\phantomsection\label{\detokenize{geometryGen:geometryGen.mergeEvent}}\pysiglinewithargsret{\sphinxcode{\sphinxupquote{geometryGen.}}\sphinxbfcode{\sphinxupquote{mergeEvent}}}{\emph{event}}{}
Create Merged wires from list of wire primitives
\begin{quote}\begin{description}
\item[{Parameters}] \leavevmode\begin{description}
\item[{\sphinxstylestrong{event}}] \leavevmode{[}list of list of ints{]}
List of wire primitives that defines a primitive event

\end{description}

\item[{Returns}] \leavevmode\begin{description}
\item[{\sphinxstylestrong{list of list of tuple{[}2{]} of ints}}] \leavevmode
Merged Event

\end{description}

\end{description}\end{quote}

\end{fulllineitems}



\section{geometryReco module}
\label{\detokenize{geometryReco:module-geometryReco}}\label{\detokenize{geometryReco:geometryreco-module}}\label{\detokenize{geometryReco::doc}}\index{geometryReco (module)}\index{checkCell() (in module geometryReco)}

\begin{fulllineitems}
\phantomsection\label{\detokenize{geometryReco:geometryReco.checkCell}}\pysiglinewithargsret{\sphinxcode{\sphinxupquote{geometryReco.}}\sphinxbfcode{\sphinxupquote{checkCell}}}{\emph{planes}, \emph{wires}}{}
Check if the wires given form a cell
\begin{quote}\begin{description}
\item[{Parameters}] \leavevmode\begin{description}
\item[{\sphinxstylestrong{planes}}] \leavevmode{[}list of PlaneInfo{]}
A list containing information for all the planes in the detector

\item[{\sphinxstylestrong{wires}}] \leavevmode{[}list of tuple{[}2{]} of int{]}
A list of wires for each plane

\end{description}

\item[{Returns}] \leavevmode\begin{description}
\item[{\sphinxstylestrong{Cell}}] \leavevmode
Cell(False, False) if it doesn’t form a cell else the generated cell

\end{description}

\end{description}\end{quote}

\end{fulllineitems}

\index{lineIntersection() (in module geometryReco)}

\begin{fulllineitems}
\phantomsection\label{\detokenize{geometryReco:geometryReco.lineIntersection}}\pysiglinewithargsret{\sphinxcode{\sphinxupquote{geometryReco.}}\sphinxbfcode{\sphinxupquote{lineIntersection}}}{\emph{line0}, \emph{line1}}{}~\begin{description}
\item[{Gives points of intersection between 2 lines}] \leavevmode
Note: Doesn’t account for parallel lines

\end{description}
\begin{quote}\begin{description}
\item[{Parameters}] \leavevmode\begin{description}
\item[{\sphinxstylestrong{line0}}] \leavevmode{[}Line{]}
First Line

\item[{\sphinxstylestrong{line1}}] \leavevmode{[}Line{]}
Second Line

\end{description}

\item[{Returns}] \leavevmode\begin{description}
\item[{\sphinxstylestrong{Point}}] \leavevmode
Intersection Point

\end{description}

\end{description}\end{quote}

\end{fulllineitems}

\index{makeLines() (in module geometryReco)}

\begin{fulllineitems}
\phantomsection\label{\detokenize{geometryReco:geometryReco.makeLines}}\pysiglinewithargsret{\sphinxcode{\sphinxupquote{geometryReco.}}\sphinxbfcode{\sphinxupquote{makeLines}}}{\emph{plane}, \emph{wire}}{}
Create lines based on a merged wire
\begin{quote}\begin{description}
\item[{Parameters}] \leavevmode\begin{description}
\item[{\sphinxstylestrong{plane}}] \leavevmode{[}PlaneInfo{]}
Plane information for the plane the wire number is for

\item[{\sphinxstylestrong{wire}}] \leavevmode{[}tuple{[}2{]} of int{]}
A merged wire

\end{description}

\item[{Returns}] \leavevmode\begin{description}
\item[{\sphinxstylestrong{list of lines}}] \leavevmode
List of lines that the boundaries of the wire correspond to

\end{description}

\end{description}\end{quote}

\end{fulllineitems}

\index{reconstructCells() (in module geometryReco)}

\begin{fulllineitems}
\phantomsection\label{\detokenize{geometryReco:geometryReco.reconstructCells}}\pysiglinewithargsret{\sphinxcode{\sphinxupquote{geometryReco.}}\sphinxbfcode{\sphinxupquote{reconstructCells}}}{\emph{planes}, \emph{event}}{}
Reconstruct cells from a merged event
\begin{quote}\begin{description}
\item[{Parameters}] \leavevmode\begin{description}
\item[{\sphinxstylestrong{planes}}] \leavevmode{[}list of PlaneInfo{]}
A list containing information for all the planes in the detector

\item[{\sphinxstylestrong{event}}] \leavevmode{[}list of list of tuple{[}2{]} of int{]}
List of merged wires in an event

\end{description}

\item[{Returns}] \leavevmode\begin{description}
\item[{\sphinxstylestrong{list of Cell}}] \leavevmode
List of cells Reconstructed from Geometric information

\end{description}

\end{description}\end{quote}

\end{fulllineitems}

\index{sortPoints() (in module geometryReco)}

\begin{fulllineitems}
\phantomsection\label{\detokenize{geometryReco:geometryReco.sortPoints}}\pysiglinewithargsret{\sphinxcode{\sphinxupquote{geometryReco.}}\sphinxbfcode{\sphinxupquote{sortPoints}}}{\emph{points}}{}
Create convex hull and sort the points of convex hull in counterClockwise order
\begin{quote}\begin{description}
\item[{Parameters}] \leavevmode\begin{description}
\item[{\sphinxstylestrong{points}}] \leavevmode{[}list of Point{]}
Input point cloud to make convex hull

\end{description}

\item[{Returns}] \leavevmode\begin{description}
\item[{\sphinxstylestrong{list of Point}}] \leavevmode
List of points that form convex hull

\end{description}

\end{description}\end{quote}

\end{fulllineitems}

\index{wireIntersection() (in module geometryReco)}

\begin{fulllineitems}
\phantomsection\label{\detokenize{geometryReco:geometryReco.wireIntersection}}\pysiglinewithargsret{\sphinxcode{\sphinxupquote{geometryReco.}}\sphinxbfcode{\sphinxupquote{wireIntersection}}}{\emph{plane0}, \emph{wire0}, \emph{plane1}, \emph{wire1}}{}
Intersection points between two merged wires in different planes
\begin{quote}\begin{description}
\item[{Parameters}] \leavevmode\begin{description}
\item[{\sphinxstylestrong{plane0}}] \leavevmode{[}PlaneInfo{]}
Plane information for the plane the wire 0 is for

\item[{\sphinxstylestrong{wire}}] \leavevmode{[}tuple{[}2{]} of int{]}
A merged wire (wire0)

\item[{\sphinxstylestrong{plane1}}] \leavevmode{[}PlaneInfo{]}
Plane information for the plane the wire 1 is for

\item[{\sphinxstylestrong{wire}}] \leavevmode{[}tuple{[}2{]} of int{]}
A merged wire (wire1)

\end{description}

\item[{Returns}] \leavevmode\begin{description}
\item[{\sphinxstylestrong{list of Point}}] \leavevmode
Intersection Points Between merged wires

\end{description}

\end{description}\end{quote}

\end{fulllineitems}



\section{draw module}
\label{\detokenize{draw:draw-module}}\label{\detokenize{draw::doc}}

\chapter{Indices and tables}
\label{\detokenize{index:indices-and-tables}}\begin{itemize}
\item {} 
\DUrole{xref,std,std-ref}{genindex}

\item {} 
\DUrole{xref,std,std-ref}{modindex}

\item {} 
\DUrole{xref,std,std-ref}{search}

\end{itemize}


\renewcommand{\indexname}{Python Module Index}
\begin{sphinxtheindex}
\def\bigletter#1{{\Large\sffamily#1}\nopagebreak\vspace{1mm}}
\bigletter{d}
\item {\sphinxstyleindexentry{dataTypes}}\sphinxstyleindexpageref{dataTypes:\detokenize{module-dataTypes}}
\indexspace
\bigletter{g}
\item {\sphinxstyleindexentry{geometryGen}}\sphinxstyleindexpageref{geometryGen:\detokenize{module-geometryGen}}
\item {\sphinxstyleindexentry{geometryReco}}\sphinxstyleindexpageref{geometryReco:\detokenize{module-geometryReco}}
\indexspace
\bigletter{u}
\item {\sphinxstyleindexentry{utilities}}\sphinxstyleindexpageref{utilities:\detokenize{module-utilities}}
\end{sphinxtheindex}

\renewcommand{\indexname}{Index}
\printindex
\end{document}